\documentclass[11pt]{jsarticle}

%tabular環境での縦書き
\usepackage{plext}
%余白設定
\setlength{\topmargin}{15pt}
\setlength{\textheight}{40\baselineskip}
\addtolength{\textheight}{\topskip}
\iftombow
	\addtolength{\topmargin}{-1in}
\else
	\addtolength{\topmargin}{-1 truein}
\fi
%図
\usepackage[dvipdfmx]{graphicx}
\usepackage{float}
\usepackage{wrapfig}
%数学の何かのフォント
\usepackage{amsmath}
\usepackage{mathpazo}
\usepackage{amssymb}
%サブキャプション
\usepackage[subrefformat=parens]{subcaption}


%sectionコマンドの編集
\renewcommand{\presectionname}{第}
\renewcommand{\postsectionname}{章}%「第~部」を「第~章」に変える
\newcommand{\newsection}[1]{\clearpage \vspace*{24zw} \section{#1} \clearpage}


%partの編集など
\usepackage{titlesec}
%\titleformat{\part}[display]{\normalfont\Huge\bfseries\filcenter}{\prepartname \thepart \postpartname}{20pt}{\Huge}

%見出しの設定
\titleformat{\section}[hang]{\normalfont\Huge\mcfamily}{\thesection}{1zw}{\filright}
\titleformat{\subsection}[hang]{\Large\mcfamily}{\thesubsection}{1zw}{}
\titleformat{\subsubsection}[hang]{\large\mcfamily}{\thesubsubsection}{1zw}{}
%目次の設定
\usepackage{titletoc}
\setcounter{tocdepth}{3}
\titlecontents{section}[1zw]{}{\thecontentslabel~~~}{}{}
\titlecontents{subsection}[2zw]{}{\thecontentslabel~~~}{}{\titlerule*{・}\makebox[2zw][r]{\thecontentspage}}
\titlecontents{subsubsection}[3zw]{}{\thecontentslabel~~~}{}{\titlerule*{・}\makebox[2zw][r]{\thecontentspage}}

%図や表のキャプションの設定
\usepackage{caption}
\captionsetup[figure]{labelsep=quad}
\captionsetup[table]{labelsep=quad}


%タイトルページ
%\title{卒論タイトル}
%\author{著者の名前}
%\date{\today}


\begin{document}

%\maketitle
%\clearpage
\tableofcontents
\clearpage

\newsection{これについて}

\subsection{背景}
背景です。
\subsubsection{背景のサブセクション}
aaa
asfijaweopghapo



jpvaewohgpuawehfaphvo

fhviarehgoiahefopqhwpaueilghap9ueosihfpa8jeyp9f
hq9wpuehf

weiuhfaowefhw

hvwe9aphgap98erog9ap8ewfhpoa

hqw9ueghaf89pwehfp9hw\\
fh9pwueaga9pwehfp

\begin{figure}[H]
	\centering
	\includegraphics[width=0.6\linewidth]{figures/tlc_pair.png}
	\caption{テスト図形 test figure}
	\label{fig:test}
\end{figure}%


jqw0@eohgap89wehf0\\
fjoas;ifeh
hp9wroi
\subsection{原理}
\newsection{内容について}
以下に表を作ります
\begin{table}
	\centering
	\caption{テスト表}
	\label{tab:test}
	\begin{tabular}{|c|c|} \hline
		あ & い \\ \hline
		asd & ewoi \\ \hline
		がぢ & kgei \\ \hline
	\end{tabular}
\end{table}
	
\subsection{提案}
\subsection{実験結果}
\subsection{考察}
\subsection{結論}
\section*{謝辞}
\section*{参考文献}
%%例:
%%		\item
%%			著者名. 書籍名. 出版場所, 出版社名, 出版年.
%%			\label{ref:example}
%%		\item
%%			著者名. ``ページタイトル''. サイト名. 
%%			\verb|URL|, 
%%			(参照 20xx-12-31).
%%			\label{ref:example2}
\begin{enumerate}
	\item
\end{enumerate}


\end{document}
