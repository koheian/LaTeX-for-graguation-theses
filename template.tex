\documentclass[11pt, a4paper]{jsarticle}

%tabular環境での縦書き
\usepackage{plext}
%余白設定
\setlength{\topmargin}{64pt}
\setlength{\textheight}{39\baselineskip}
\addtolength{\textheight}{\topskip}
\iftombow
	\addtolength{\topmargin}{-1in}
\else
	\addtolength{\topmargin}{-1 truein}
\fi
%図
\usepackage[dvipdfmx]{graphicx}
\usepackage{float}
%\usepackage{wrapfig}
%数学の何かのフォント
\usepackage{amsmath}
\usepackage{mathpazo}
\usepackage{amssymb}
%式番号に章番号もつける
\numberwithin{equation}{section}
%サブキャプション
\usepackage[subrefformat=parens]{subcaption}

%テキスト部分の長さ設定
\setlength{\textheight}{566pt}
%ヘッダの編集
\usepackage{fancyhdr}
\pagestyle{fancy}
\rhead{}
\lhead{}
\renewcommand{\headrulewidth}{0.5pt}
\setlength{\headsep}{28pt}
%フッタの編集
\cfoot{--~~\thepage~~--}
\setlength{\footskip}{33pt}



%sectionコマンドの編集
\newcommand{\newsection}[1]{\clearpage \vspace*{20zw} \section{#1} \clearpage}


%partの編集など
\usepackage{titlesec}
%\titleformat{\part}[display]{\normalfont\Huge\bfseries\filcenter}{\prepartname \thepart \postpartname}{20pt}{\Huge}

%見出しの設定
\titleformat{\section}[hang]{\normalfont\Huge\mcfamily}{第\thesection{}章}{1zw}{\filright}%「第~章」という表示に変える
\titleformat{\subsection}[hang]{\Large\mcfamily}{\thesubsection}{1zw}{}
\titleformat{\subsubsection}[hang]{\large\mcfamily}{\thesubsubsection}{1zw}{}
%目次の設定
\usepackage{titletoc}
\setcounter{tocdepth}{3}
\titlecontents{section}[1zw]{}{第\thecontentslabel{}章~~~}{}{}
\titlecontents{subsection}[2zw]{}{\thecontentslabel~~~}{}{\titlerule*{・}\makebox[2zw][r]{\thecontentspage}}
\titlecontents{subsubsection}[3zw]{}{\thecontentslabel~~~}{}{\titlerule*{・}\makebox[2zw][r]{\thecontentspage}}

%図や表のキャプションの設定
\usepackage{caption}
\captionsetup[figure]{labelsep=quad}
\captionsetup[table]{labelsep=quad}
  
%行送り
%\fontsize{11pt}{16.4pt}\selectfont
\renewcommand{\baselinestretch}{0.9}


%タイトルページ
%\title{卒論タイトル}
%\author{著者の名前}
%\date{\today}

%\usepackage{layout}

\begin{document}
%\layout

%\maketitle
%\clearpage
\pagenumbering{roman}
\tableofcontents
\clearpage
\pagenumbering{arabic}

\newsection{これについて}

\subsection{本研究の背景及び目的}
カメラカメラカメラカメラカメラカメラカメラカメラカメラカメラカメラカメラカメラカメラカメラカメラカメラカメラカメラカメラカメラカメラカメラカメラカメラカメラカメラカメラカメラカメラカメラカメラカメラカメラカメラカメラカメラカメラカメラカメラカメラカメラカメラカメラカメラカメラカメラカメラカメラカメラカメラカメラカメラカメラカメラカメラカメラカメラカメラカメラカメラカメラカメラカメラカメラカメラカメラカメラカメラカメラカメラカメラカメラカメラカメラカメラカメラカメラカメラカメラカメラカメラカメラカメラカメラカメラカメラカメラカメラカメラカメラカメラカメラカメラカメラカメラカメラカメラカメラカメラカメラカメラカメラカメラカメラカメラカメラカメラカメラカメラカメラカメラカメラカメラカメラカメラカメラカメラカメラカメラカメラカメラカメラカメラカメラカメラカメラカメラカメラカメラカメラカメラカメラカメラカメラカメラカメラカメラカメラカメラカメラカメラカメラカメラカメラカメラカメラカメラカメラカメラカメラカメラカメラカメラカメラカメラカメラカメラカメラカメラカメラカメラカメラカメラカメラカメラカメラカメラカメラカメラカメラカメラカメラカメラカメラカメラカメラカメラカメラカメラカメラカメラカメラカメラカメラカメラカメラカメラカメラカメラカメラカメラカメラカメラカメラカメラカメラカメラカメラカメラカメラカメラカメラカメラカメラカメラカメラカメラカメラカメラカメラカメラカメラカメラカメラカメラカメラカメラカメラカメラカメラカメラカメラカメラカメラカメラカメラカメラカメラカメラカメラカメラカメラカメラカメラカメラカメラカメラカメラカメラカメラカメラメラカメラカメラカメラカメラカメラカメラカメラカメラカメラカメラカメラカメラカメラカメラカメラカメラカメラカメラカメラカメラカメラカメラカメラカメラカメラカメラカメラカメラカメラカメラカメラカメラカメラカメラカメラカメラカメラカメラカメラカメラカメラカメラカメラカメラカメラカメラカメラカメラカメラカメラカメラカメラカメラカメラカメラカメラカメラカメラカメラカメラカメラカメラカメラカメラカメラカメラカメラカメラカメラカメラカメラカメラカメラカメラカメラカメラカメラカメラカメラカメラカメラカメラカメラカメラカメラカメラカメラカメラカメラカメラカメラカメラカメラカメラカメラカメラカメラカメラカメラカメラカメラカメラカメラカメラカメラカメラカメラカメラカメラカメラカメラカメラカメラカメラカメラカメラカメラカメラカメラカメラカメラカメラカメラメラカメラカメラカメラカメラカメラカメラカメラカメラカメラカメラカメラカメラカメラカメラカメラカメラカメラカメラカメラカメラカメラカメラカメラカメラカメラカメラカメラカメラカメラカメラカメラカメラカメラカメラカメラカメラカメラカメラカメラカメラカメラカメラカメラカメラカメラカメラカメラカメラカメラカメラカメラカメラカメラカメラカメラカメラカメラカメラカメラカメラカメラカメラカメラカメラカメラカメラカメラカメラカメラカメラカメラカメラカメラカメラカメラカメラカメラカメラカメラカメラカメラカメラカメラカメラカメラカメラカメラカメラカメラカメラカメラカメラカメラカメラカメラカメラカメラカメラカメラカメラカメラカメラカメラカメラカメラカメラカメラカメラカメラカメラカメラカメラカメラカメラカメラカメラカメラカメラカメラカメラカメラカメラカメラメラカメラカメラカメラカメラカメラカメラカメラカメラカメラカメラカメラカメラカメラカメラカメラカメラカメラカメラカメラカメラカメラカメラカメラカメラカメラカメラカメラカメラカメラカメラカメラカメラカメラカメラカメラカメラカメラカメラカメラカメラカメラカメラカメラカメラカメラカメラカメラカメラカメラカメラカメラカメラカメラカメラカメラカメラカメラカメラカメラカメラカメラカメラカメラカメラカメラカメラカメラカメラカメラカメラカメラカメラカメラカメラカメラカメラカメラカメラカメラカメラカメラカメラカメラカメラカメラカメラカメラカメラカメラカメラカメラカメラカメラカメラカメラカメラカメラカメラカメラカメラカメラカメラカメラカメラカメラカメラカメラカメラカメラカメラカメラカメラカメラカメラカメラカメラカメラカメラカメラカメラカメラカメラカメラ


\begin{equation}
a = b \label{eq:test}
\end{equation}

テストの式は式(\ref{eq:test})です。

カメラカメラカメラカメラカメラカメラカメラカメラカメラカメラカメラカメラカメラカメラカメラカメラカメラカメラカメラカメラカメラカメラカメラカメラカメラカメラカメラカメラカメラカメラカメラカメラカメラカメラカメラカメラカメラカメラカメラカメラカメラカメラカメラカメラカメラカメラカメラカメラカメラカメラカメラカメラカメラカメラカメラカメラカメラカメラカメラカメラカメラカメラカメラカメラカメラカメラカメラカメラカメラカメラカメラカメラカメラカメラカメラカメラカメラカメラカメラカメラカメラカメラカメラカメラカメラカメラカメラカメラカメラカメラカメラカメラカメラカメラカメラカメラカメラカメラカメラカメラカメラカメラカメラ

\subsubsection{背景のサブセクション}
aaa
asfijaweopghapo



jpvaewohgpuawehfaphvo

fhviarehgoiahefopqhwpaueilghap9ueosihfpa8jeyp9f
hq9wpuehf

weiuhfaowefhw

hvwe9aphgap98erog9ap8ewfhpoa

hqw9ueghaf89pwehfp9hw\\
fh9pwueaga9pwehfp

\begin{figure}[H]
	\centering
	\includegraphics[width=0.6\linewidth]{test.png}
	\caption{テスト図形 test figure}
	\label{fig:test}
\end{figure}%


jqw0@eohgap89wehf0\\
fjoas;ifeh
hp9wroi
\subsection{原理}
\newsection{内容について}
以下に表を作ります
\begin{table}
	\centering
	\caption{テスト表}
	\label{tab:test}
	\begin{tabular}{|c|c|} \hline
		あ & い \\ \hline
		asd & ewoi \\ \hline
		がぢ & kgei \\ \hline
	\end{tabular}
\end{table}


2こめの式
\begin{equation}
asl;gtueoi
\end{equation}
\begin{equation}
aedd=gtueoi
\end{equation}
	
\subsection{提案}
箇条書き
\begin{enumerate}
	\item その壱
	\item その弐
\end{enumerate}

\subsection{実験結果}
\subsection{考察}
\subsection{結論}
\section*{謝辞}
\section*{参考文献}
%%例:
%%		\item
%%			著者名. 書籍名. 出版場所, 出版社名, 出版年.
%%			\label{ref:example}
%%		\item
%%			著者名. ``ページタイトル''. サイト名. 
%%			\verb|URL|, 
%%			(参照 20xx-12-31).
%%			\label{ref:example2}

%箇条書きの番号に括弧をつける
\renewcommand{\labelenumi}{[\arabic{enumi}]}
\begin{enumerate}
	\item 参考にした本
\end{enumerate}


\end{document}
